\chapter{Wstęp}
\label{ch:intro}

\section{Informacje ogólne}
\label{sec:intro-general}

Zazwyczaj w tym miejscu wszystkie podręczniki opisują takie informacje jak czym
są gry RPG, podział na graczy i GMa, czym zajmuje się GM albo co to znaczy
\dice{2d6+3}. Zakładam jednak, że czytelnik ma już doświadczenie z grami RPG,
więc nie potrzebuje wyjaśnienia tak prostych i elementarnych rzeczy.

Zamiast tego wyjaśnię w dużym skrócie czemu powstał ten system. Opiszę również
ogólny zarys świata i jakie wyzwania mogą czekać graczy w zależności od tego,
jakie postaci przygotują.

\section{Czemu ten podręcznik istnieje}
\label{sec:intro-why}

Grałem w różne systemy RPG, takie jak Warhammer Fantasy RP czy D\&D 5e.
Widziałem i czytałem również inne systemy, między innymi Call of Cthulhu i
Pathfinder 2e. Każdy system ma swoje mocne i słabe strony.

Opisany w tym podręczniku system jest swego rodzaju mieszanką wielu różnych
systemów, które w moim odczuciu bardziej pasuje do mrocznego świata Kamorii.
Uwzględnia on podział rasowy, opisany w sekcji~\ref{sec:racism}. Modyfikuje on
również atrybuty postaci i dostępne klasy, opisane dokładniej w
rozdziale~\ref{ch:character}

\section{Świat Kamorii}
\label{sec:intro-world}

Poniższy opis jest bardzo powierzchowny i nie zawiera większości ważnych
informacji. Służy on tylko do pokazania jaki "vibe" ma Kamoria. Oczywiście,
zasady z tego podręcznika można używać w bardziej tradycyjnych światach, takich
jak Forgotten Realms znanego z D\&D. Więcej informacji można znaleźć w
rozdziale~\ref{ch:world}

Kamoria jest światem, w którym istnieje tylko jeden kraj. Zajmuje on część
jedynego kontynentu na świecie. Z północnego wschodu znajdują się Góry
Graniczne, z za których przychodzą różne potwory. Reszta kraju ograniczona jest
oceanem.

Przez kraj płyną kanały z krystalicznie czystą wodą, zasilanych przez jezioro,
na środku którego znajduje stolica kraju. Są to pozostałości po Dawnej
Cywilizacji, która zostawiła również dziesiątki, jeśli nie setki ruin,
podziemnych placówek czy dawnych pól bitwy. Większość z tych miejsc do tej pory
nie została zbadana i nadal jest strzeżona przez przedziwne potwory.

Społeczeństwo podzielone jest na Rasy Wyższe, Niższe i Zwierzęce. Rasy Wyższe to
ludzie, krasnoludy i elfy; Niższe to Gnomy, Firbolgi, Pół Elfy, Pół Orki,
Halflingi i Orki. Tabaxi, Kenku, Lizardfolk i wszystkie pozostałe rasy bazujące
na zwierzętach to Rasy Zwierzęce. Pozostałe rasy uznawane są za potwory -
Tiefling jest traktowany tak samo jak Goblin.

W większości przypadków interakcje między grupami ras ograniczają się do
wydawania rozkazów. Rasy Wyższe to arystokracja, Niższe to mieszczaństwo i
robotnicy, a Zwierzęce są traktowane jak niewolnicy i używane do zaspokajania
Ras Wyższych.

Jedynym miejscem, gdzie podział zostaje zniesiony jest Gildia. Do Gildii może
dołączyć każdy, niezależnie od rasy. Chociaż oficjalnie każdy członek Gildii
jest sobie równy i jest poza standardowym podziałem, nie każdy respektuje tą
zasadę. Często zdarza się, że na przykład Elf odmawia współpracowania z Tabaxi,
a podczas misji niektórzy członkowie Ras Wyższych traktują członków Gildii z Ras
Zwierzęcych jak swoich niewolników.
