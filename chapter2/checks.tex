\section{Test umiejętności}

Wykonywanie testów umiejętności odbywa się poprzez wykonanie rzutu \dice{d20} i
dodanie bonusu wynikającego z odpowiedniego atrybutu tyle razy ile wskazuje
liczba przy atrybucie.

\begin{tcolorbox}
	{\large\bfseries Przykład}\\
Bartek chce przebiec zamrożone jezioro. Musi wtedy wykonać test
Biegania. Jego postać ma modyfikator Siły równy \( +1 \), modyfikator
Wytrzymałości \( +0 \) i modyfikator Zręczności \( +4 \). Wybiera więc zrobienie
testy Biegania za pomocą Zręczności. Wynik jego testu to \dice{1d20 + 4}.

Tomek również chce przebiec to jezioro, jednak jego modyfikatory Siły,
Zręczności i Wytrzymałości są równe \( +1 \), wybiera więc użycie Siły. Jego
test Biegania ma wynik \dice{1d20 + 2 * 1}, gdyż siła ma modyfikator
umiejętności równy \( 2 \).
\end{tcolorbox}

W uzasadnionych przypadkach Mistrz Gry może powiedzieć, że test umiejętności
należy wykonać za pomocą konkretnego atrybutu. Mistrz Gry decyduje wtedy
również jaki modyfikator atrybutu należy użyć: wskazany w podręczniku, czy \( 1
\).

\begin{tcolorbox}
	{\large\bfseries Przykład}\\
	Bartek chce naprawić zegarek, który znalazł podczas ostatniej misji. Wymaga
	to pewnej ręki i precyzji, więc MG nakazuje mu wykonanie testu Tworzenia
	używając Zręczności.

	W tym czasie Tomek chce wykuć nowy miecz. Kucie stali wymaga sporej siły,
	więc test Tworzenia wykonywany jest z użyciem Siły.
\end{tcolorbox}
