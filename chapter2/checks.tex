\section{Test umiejętności}

Wykonywanie testów umiejętności odbywa się poprzez wykonanie rzutu \dice{d10}  i
wykonanie kolejnych rzutów Kości Umiejętności zależnych  od  poziomu  i  Stopnia
Treningu.  Wszystkie  kości  eksplodują,  wliczając  w	to	Kości  Umiejętności,
których  ilość	i  wielkość  zapisane  są	w	tabeli~\nameref{tab:skill-dice}.
Jednakże wyrzucenie  \dice{Nat	1}	na	Kości  Umiejętności  nie  liczy  się  do
ostatecznego wyniku, a \dice{Nat 1} na \dice{d10} oznacza automatyczną	porażkę,
często z konsekwencjami.


\begin{boxed}{Przykład}
	Bartek chce  przebiec  po  zamrożonej  rzece.	Wymaga	to	zrobienia  testu
	Biegania.  Jego Zręczność to $15$, Siła to	$11$  i  Wytrzymałość  to  $13$,
	wybiera więc użycie Zręczności.  Dodatkowo z racji swojego Pochodzenia	jest
	Ekspertem w Bieganiu.  Jego test umiejętności wymaga więc rzutu \dice{1d10 +
	3d8}, z czego \dice{3d8} to Kości  Umiejętności  za  jego  poziom  Treningu.

	Adam chce przebiec po tej samej rzece, jednak jego Zręczność to 7, a Siła  i
	Wytrzymałość są równe 8 i jest tylko  Wytrenowany  w  Bieganiu.   Jego	test
	wymaga rzutu \dice{d10 - 2d4} niezależnie od tego, który  Atrybut  wybierze.

	Mistrz Gry ustawił poziom trudności  tego  testu  na  14.	Bartek	wyrzucił
	kolejno $6, 2, 3\ \text{i}\ 3$, czyli razem 14, więc  zdał	test,  choć  był
    bardzo blisko porażki. Adam wyrzucił $10, 1\ \text{i}\ 2$. Z racji że kostki
    eksplodują przy maksymalnym wyniki, rzuca \dice{d10} jeszcze raz. Jako wynik
	otrzymał $8$, a ponieważ wyrzucenie jedynki na Kości  Umiejętności	sprawia,
    że kostka nie liczy się do ostatecznego wyniku. Wynik testu dla Adama wynosi
	$  10  +  8  -	2  =  16$,	czyli  o  dwa  oczka  więcej  niż	potrzebował.
\end{boxed}

W uzasadnionych przypadkach Mistrz Gry może  powiedzieć,  że  test	umiejętności
należy wykonać za  pomocą  konkretnego  atrybutu.

\begin{boxed}{Przykład}
	Bartek chce naprawić zegarek, który znalazł podczas ostatniej misji.  Wymaga
	to pewnej ręki i precyzji, więc MG nakazuje  mu  wykonanie	testu  Tworzenia
    używając Zręczności.

	W tym czasie Tomek chce wykuć nowy miecz.  Kucie stali wymaga  sporej  siły,
    więc test Tworzenia wykonywany jest z użyciem Siły.
\end{boxed}

