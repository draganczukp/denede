\section{Modyfikator Treningu}

Oprócz samej wartości Atrybutu, postaci mogą trenować konkretne Umiejętności.
Postaci zaczynają z większością Umiejętności na poziomie Niewytrenowanym. W
zależności od Rasy, Przeszłości i Klasy postaci zaczynają z niektórymi
Umiejętnościami na poziomie Wytrenowanym, lub nawet Eksperckim.

Dostępne poziomy Treningu to:
\begin{itemize}
	\item Niewytrenowany
	\item Wytrenowany
	\item Ekspert
	\item Mistrz
	\item Legenda
\end{itemize}

Zwiększenie poziomu Treningu dla Umiejętności jest możliwe poprzez spędzenie
czasu w grze na trening. Dodatkowo niektóre Zdolności pozwalają na zwiększenie
poziomu konkretnych Umiejętności lub nauczenie się ich na poziom Wytrenowany.

Tabela~\nameref{tab:skill-dice} pokazuje jakie kostki należy używać przy
wykonywaniu testów Umiejętności w zależności od wartości odpowiadającego
atrybutu i poziomu Treningu dla danej Umiejętności.

\begin{table}[h]
\centering
\resizebox{\textwidth}{!}{%
\begin{tabular}{@{}c|ccccc@{}}
\toprule
Wartość & Niewytrenowany & Wytrenowany & Ekspert & Mistrz & Legenda \\ \midrule
1-2     & \multicolumn{5}{c}{-5d4}                                  \\
3-4     & \multicolumn{5}{c}{-4d4}                                  \\
5-6     & \multicolumn{5}{c}{-3d4}                                  \\
7-8     & \multicolumn{5}{c}{-2d4}                                  \\
9-10    & \multicolumn{5}{c}{-1d4}                                  \\
11-12   & 0              & 1d4         & 1d6     & 1d8    & 1d10    \\
13-14   & 1d4            & 1d6         & 1d8     & d12    & 2d8     \\
15-16   & 2d4            & d12         & 2d10    & 2d12   & 2d20    \\
17-18   & 3d4            & 2d12        & 4d8     & 2d20   & 4d12    \\
19-20   & 4d4            & 4d6         & 3d10    & 3d12   & 4d10    \\ \bottomrule
\end{tabular}%
}
\caption{Zestawienie Kości Umiejętności w zależności od wartości Atrybutu i poziomu Treningu w danej Umiejętności}
\label{tab:skill-dice}
\end{table}
