\section{Zasady ogólne}
\subsection{Resurrection}

Rezurekcje nie są idealne.	Jeśli dusza opuściła ciało, może jej być  ciężko  do
niego wrócić, albo może dany bóg woli, by dusza została z nim zamiast wrócić  do
świata żywych.

Przy każdej próbie wskrzeszenia danej osoby wymagana jest pomoc boska.	 Postać,
która próbuje wskrzesić inną postać zaczyna rytuał, który ma na  celu  przekonać
boga by zgodził się pomóc.	Pozostałe postaci mogą również brać  w	nim  udział.
Opisują wtedy w jaki sposób starają się przekonać  boga,  a  następnie	wykonują
odpowiedni test umiejętności.

Gdy wszyscy, którzy chcą pomóc w rytuale zrobią zadeklarują jak pomagają,  osoba
prowadząca rytuał wykonuje rzut d20 i dodaje do niego  modyfikator	inteligencji
oraz modyfikator proficiency.	Rzut  ten  musi  przebić  DC  10,  jednak  każde
wskrzeszenie	danej	 postaci	zwiększa	jej    osobiste    DC	 o	  2.

Za każdy sukces od postaci pomagających zmniejsza DC o 1 lub  o  3	w  przypadku
Nat  20.   Każda  porażka	zwiększa   je	o	1	lub   o   2   dla	Nat   1.

Jeśli rzut nie uda się, wskrzeszenie nie udaje się.  Oznacza to, że bóg  odmawia
pomocy, lub że dusza jest  zbyt  zmęczona  ciągłymi  wskrzeszeniami  by  wrócić.

\subsection{Nowy poziom}
Zwiększenie poziomu wymaga długiego odpoczynku, najlepiej w bezpiecznych
warunkach takich jak dom lub karczma.

Podczas rzucania w celu zwiększenia maksymalnych punktów zdrowia, jeśli na
kostce wypadnie 1, gracz może rzucać jeszcze raz.

Dodatkowo jeśli gracz nie jest zadowolony ze swojego rzutu, DM może również
rzucić, ale gracz nie zna wyniku, a DM nie przerzuca jedynek. Gracz może wybrać,
czy woli swój rzut czy rzut DMa, którego nie zna. Decyzja jest nieodwołalna.
