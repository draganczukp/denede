\section{Zasady do walki}

\subsection{Krytyczne trafienia}

Zwykłe	krytyczne  trafienia  mogą	nie  zadać	większych  obrażeń	niż   zwykłe
trafienia, jeśli oba  rzuty  dadzą	niski  wynik.	Powoduje  to,  że  trafienia
krytyczne są rozczarowujące i niweluje to ekscytacje spowodowaną uśmiechem losu.

Lepszą zasadą na obrażenia krytyczne  jest	dodawanie  maksymalnych  obrażeń  do
wyrzuconej wartości. Dla przykładu: miecz który normalnie zadaje 1d8 obrażeń, po
wyrzuceniu krytycznego trafienia zadaje $1d8 + 8$ obrażeń.

Krytyczne trafienia wpływają również na wszelkie modyfikatory obrażeń, takie jak
Sneak Attack czy Divine Smite. Przykładowo: Rogue na 7 poziomie trafia z lekkiej
kuszy za 1d8 +	4d6  obrażeń;  przy  trafieniu	krytycznym	wyrzucone  obrażenia
zwiększane są o $8 + 4*6 = 32$,  czyli	maksymalną	wartość  kości	od	obrażeń.

\subsubsection{Podwójne trafienie krytyczne}

Choć jest to mało prawdopodobne, gracz posiadający	ułatwienie	lub  utrudnienie
może na obu kostkach d20 wyrzucić trafienie krytyczne. W tym przypadku obrażenia
nie są rzucane.  Wynikowe obrażenia równe są dwukrotności maksymalnych	obrażeń.

Dla wyżej wymienionego przykładu,  jeśli  gracz  wyrzuciłby  podwójne  trafienie
krytyczne, zadane obrażenia byłyby równe 64.

\subsection{Krytyczne porażki}

O ile krytyczne sukcesy (Nat 20) oznaczają	wyjątkowy  uśmiech	losu,  krytyczne
porażki oznaczają, że los sprzeciwia się postaci.  Oznacza to, że Nat 1  powinno
mieć konsekwencje, które wypisane są w	poniższych	tabelkach  w  zależności  od
rodzaju ataku.

Gracz  rzuca  d20,	a  wynik  rzutu  decyduje,	jak  poważne  są   konsekwencje.
Tabela~\ref{tab:crit-fail} pokazuje wyniki rzutów w zależności od rodzaju ataku.

\newpage
\begin{table}[h]
\centering
\begin{tabular}{c M{.45\textwidth} M{.45\textwidth}}
\toprule
\textbf{d20}   & Melee                  & Zasięgowy                \\ \midrule
\textbf{20}    & \multicolumn{2}{c}{Ponowienie ataku}              \\
\textbf{19-15} & \multicolumn{2}{c}{Nic się nie dzieje}            \\
\textbf{14-10} &
  Postać upuszcza trzymaną broń &
  W   przypadku   broni,   broń   psuje   się	(zerwana   cięciwa	 itp.)	  \\
\textbf{9-5} &
  Ponowienie ataku, jednak celem jest losowa postać w zasięgu 5ft od atakującego &
  Ponowienie ataku, celem jest losowa osoba w zasięgu w 60°; może to być przeciwnik lub sojusznik \\
\textbf{4-2} &
  Jeśli  w	zasięgu   5ft	jest   jakiś   sojusznik,	jest   on	trafiany   &
  Trafiany	jest  sojusznik,   którego	 trafienie	 ma   najwięcej   sensu   \\
\textbf{1} & \multicolumn{2}{c}{Tragedia.  Pełna dowolność DMa}  \\  \bottomrule
\end{tabular}
\caption{Tabela konsekwencji krytycznej porażki}
\label{tab:crit-fail}
\end{table}

\subsection{Inicjatywa}
Wyrzucenie naturalnej 20 podczas rzutu na inicjatywę  oznacza,	że	postać	jest
wyjątkowo  gotowa  do  walki.	Postać	trafia	wtedy  na  sam	początek  listy,
niezależnie  od  wszystkich  modyfikatorów.   Dodatkowo  pierwszy  atak  podczas
pierwszej tury postaci ma Advantage.

Analogicznie, wyrzucenie naturalnej 1 postać jest  ostatnia  na  liście,  a  jej
pierwszy atak lub pierwszy Saving Throw ma Disadvantage.

\subsection{Death Save}
Death Save są widoczne tylko dla DMa i gracza. Reszta graczy nie zna ich wyniku.
Dzięki	temu  nie  powstają  sytuacje  w  których  gracze  wolą  skupić  się  na
przeciwniku zamiast pomagać sobie nawzajem, bo przecież ma dwa	sukcesy  i	zero
porażek.

\subsection{Koszty akcji}
Niektóre rzeczy kosztują akcję, choć nie powinny.	Inne  nic  nie	kosztują,  a
powinny.

\subsubsection{Potiony}
Wypicie potiona samemu kosztuje  bonusową  akcję,  natomiast  zaaplikowanie  jej
innemu graczowi kosztuje pełną akcję.

\subsubsection{Rozmowy}
Jeżeli gracz chce coś powiedzieć tylko w celach RP, czyli np. krzyczenie z bólu,
albo wyzywanie przeciwnika nie kosztuje żadnych  akcji	i  może  być  wykonywane
podczas tur innych postaci.

Rozmowy, którymi  gracz  chce  coś	osiągnąć,  np.	 dowiedzieć  się  czegoś  od
przeciwnika albo przekonać wroga do poddania się, kosztują całą akcję.

Jeżeli gracze chcą porozmawiać ze sobą podczas walki, kosztuje to pełną turę za
każde 10 sekund rozmowy. Po upływie tych dziesięciu sekund następuje kolejna
tura. Jeśli jest to tura gracza, może ją przeznaczyć na kolejne 10 sekund
rozmowy. W turze przeciwnika gracze mogą nadal rozmawiać, jednak przeciwnicy
będą atakować i poruszać się normalnie.

\subsection{Sztylet w drugiej ręce}
Jeżeli gracz trzyma sztylet w drugiej dłoni, może wykorzystać bonusową akcję by
nim rzucić. Jego zasięg jest wtedy zmniejszony do 15/40.

\subsection{Aktywna obrona}
Zamiast używać statycznej klasy pancerza, gracze mogą próbować bronić się przed
atakami.

Jeśli atak przeciwnika nie przebija zwykłej AC postaci, jest to uznawane za
nietrafiony atak. Może przeciwnik potknął się, albo postać zrobiła zwykły unik
lub zablokowała tarczą czy broń po prostu ześlizgnęła się po zbroi.

Gdy AC gracza zostaje przebite, może on próbować się obronić, jednak musi to
zrobić zanim pozna obrażenia. Obrona przeciwko Nat 20 jest niemożliwa. Udana
obrona oznacza na przykład, że postać odskoczyła  w ostatnim momencie albo
przeciwnik akurat trafił w twardszy element pancerza.

W celu obrony przed atakiem gracz rzuca d20 i dodaje do wyniku swoją klasę
pancerze pomniejszoną o 10. Jeśli wynik jest większy od wartości ataku, obrona
jest udana.

\subsubsection{Ułatwienia i utrudnienia ataku}
Jeżeli przeciwnik miał ułatwienie na swoim ataku, gracz ma utrudnienie na
obronie. Analogicznie, utrudnienie na ataku oznacza ułatwienie na obronie.

\subsubsection{Krytyczne sukcesy i porażki}
Jeśli gracz wyrzuci Nat 20, obrona automatycznie jest udana. Dodatkowo gracz
może próbować "oddać" przeciwnikowi poprzez wykonanie standardowego ataku. Atak
zawsze ma utrudnienie (disadvantage) i nawet jeśli gracz ma ułatwienie z
dowolnego źródła, nie anuluje w ten sposób tego utrudnienia. Można o tym myśleć
jak o idealnym zablokowaniu lub uniknięciu przeciwnika, przez co traci on
równowagę i odsłania się do ataku, którego aż grzech nie wykorzystać.

W przypadku wyrzucenia Nat 1 atak przeciwnika staje się krytycznym trafieniem.
Fabularnie gracz próbuje zablokować lub uniknąć przeciwnika, jednak źle to
wyliczył w czasie, więc odsłonił przypadkiem nieopancerzony bok.

\subsection{Poważne Obrażenia}

\begin{boxed}{Work in Progress}
  Poważne obrażenia nie są jeszcze gotowe do gry. Wszystko, co jest napisane w tej
  sekcji jest tylko poglądowe i może zmienić się w każdej chwili.
\end{boxed}

